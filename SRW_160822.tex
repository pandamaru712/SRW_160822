\documentclass[technicalreport]{ieicej}
%\documentclass[technicalreport,usejistfm]{ieicej}
\usepackage[dvipdfmx]{graphicx}
\usepackage{latexsym}
%\usepackage[fleqn]{amsmath}
\usepackage{amsmath}
\usepackage{amssymb}
\usepackage{multirow,eepic}
\usepackage{cite}
\usepackage{mediabb}
\usepackage{url}
\usepackage{comment}
\usepackage{epsfig}
\usepackage{subfig}
\setlength{\oddsidemargin}{-9mm}
\setlength{\evensidemargin}{-9mm}
\setlength{\topmargin}{-4mm}
%\renewcommand{\topfraction}{1.0}
%\renewcommand{\bottomfraction}{1.0}
%\renewcommand{\dbltopfraction}{1.0}
%\renewcommand{\textfraction}{0.01}
%\renewcommand{\floatpagefraction}{1.0}
%\renewcommand{\dblfloatpagefraction}{1.0}
\setcounter{topnumber}{5}
\setcounter{bottomnumber}{5}
\setcounter{totalnumber}{10}

\newcommand{\sij}{(i,\ j)}
\newcommand{\mN}{{\mathcal N}}
\newcommand{\pij}{p^{(i,\ j)}}
\newcommand{\rd}{r^{\sij}_{\rm d}}
\newcommand{\ru}{r^{\sij}_{\rm u}}
\newcommand{\rij}{r^{\sij}}
\newcommand{\etau}{\eta_{\rm u}^{(j)}}

\jtitle{全二重通信無線LANにおける公平性とQoSの改善}
%\jsubtitle{}
\etitle{Fairness and QoS improvement for in-band full-duplex WLANs}
%\esubtitle{}
 \alignorder=4
 \breakauthorline{4}

\authorlist{%
 \authorentry[info16@imc.cce.i.kyoto-u.ac.jp]{飯田 直人}{Naoto Iida}{^1}
 \authorentry{西尾 理志}{Takayuki NISHIO}{^1}
 \authorentry{守倉 正博}{Masahiro MORIKURA}{^1}
 \authorentry{山本 高至}{Koji YAMAMOTO}{^1}
 \authorentry{鍋谷 寿久}{Toshihisa Nabetani}{^2}
 \authorentry{青木 亜秀}{Tsuguhide Aoki}{^2}
 }

\affiliate[^1]{京都大学 大学院情報学研究科\hskip1zw
  〒606-8501 京都市左京区吉田本町}
 {Graduate School of Informatics, Kyoto University\hskip1em
  Yoshida-honmachi, Sakyo-ku, Kyoto,
  606-8501 Japan}
\affiliate[^2]{株式会社東芝 研究開発センター\hskip 1zw 〒212-8582 神奈川県川崎市幸区小向東芝町1}
  {Corporate Research \& Development Center, TOSHIBA Corporation\hskip1em
   1 Komukaitoshiba-cho, Saiwai-ku, Kawasaki-shi, Kanagawa 212-8582, Japan}

\begin{document}
\begin{jabstract}
	全二重通信無線LANでは送信と受信を同時に同じ周波数帯で行うため,
	1台のAP(Access Point)とその左右に配置された2台のSTA(station) A,Bによる一方向全二重通信の場合,
	APの受信信号にAPの送信信号が影響を与える自己干渉と,
	STA Aの送信信号がSTA Bの受信信号に干渉を与えるユーザ間干渉が問題となる.
	自己干渉は自己干渉除去技術によって低減可能であるが,ユーザ間干渉は除去できない.
	そのため,ユーザ間干渉の影響を小さくするため,全二重通信に参加する2台のSTAの組み合わせをユーザ間干渉を考慮して決定する手法が検討されている.
	しかし,従来の手法ではシステムスループットを最大化するために,
	自己干渉やユーザ間干渉の量に応じて変動する過去の受信成功確率やスループットをもとにSTAの組み合わせを決定するため,
	STA間の公平性やアプリケーションサービスの要求の違いを考慮できていない.
	特に,遅延が小さいことが望ましいアプリケーションサービスを用いているSTAであっても,干渉が大きければ送信機会を得にくくなる.
	本稿ではこの問題解決のために,従来研究で定式化されたSTAの組み合わせを決定するための最適化問題に変更を加え,
	新たな組み合わせ決定手法を提案する.
	まず,各STAが送信を待機している時間を考慮にいれることでSTA間での公平性を改善する.
	さらに,遅延が小さいことが望ましいアプリケーションサービスを用いるSTAの送信機会を増加させ,QoS(Quality of Service)を向上させる手法を提案する.
	さらに,提案手法による公平性の改善効果と低遅延を要求するSTAのQoSの向上効果を計算機シミュレーションにより明らかにする.
\end{jabstract}
\begin{jkeyword}
全二重通信無線LAN,ユーザ間干渉,最適化
\end{jkeyword}
\begin{eabstract}
	Wireless devices of in-band full-duplex wireless local area networks (WLANs) can transmit and receive at the same time and the same frequency channel.
	In uni-directional full-duplex WLANs by an access point (AP), station (STAs) A and B,
	self-interference and inter-user interference become a big problem.
	The self-interference is an interference in the received signal of AP caused by own transmission signal,
	and the inter-user interference is an interference in STA B caused by STA A.
	The self-interference can be canceled sufficiently by the cutting edg cancellation technology, but the inter-user interference cannot be canceled.
	Conventional methods had already proposed schemes to select a pair of STAs to communicate in the full-duplex system considering influences of the inter-user interferences.
	However, since these schemes select STAs based on only the past full-duplex transmission success probability or throughput, they cannot take into account fairness between STAs and application services which require short delay.
	In this paper, we modify optimization problem of the conventional method.
	We propose a scheme to improve the fairness and quality of service (QoS).
	Simulation results show that the proposed methods improve the fairness and QoS.

\end{eabstract}
\begin{ekeyword}
full-duplex wireless LAN, inter-user interference, optimization
\end{ekeyword}

\titlepagebaselinestretch{0.92}

\maketitle

\section{はじめに}
	近年,無線LAN(Local Area Network)が急速に普及し,急増するトラヒックにより2.4\,GHz帯は逼迫しており,
	近い将来5\,GHz帯も同様の状態になることで, スループットの低下が問題となる.
	そのような状況において,無線LANシステムの大容量化が望まれる.
	大容量化を実現する方法の一つとして,送信と受信を同時に同じ周波数帯で行う全二重通信無線LANが有望である.
	全二重通信無線LANでは送信と受信を同時に同じ周波数帯で行うため,理想的には周波数利用効率を2倍にすることができる.
	この全二重通信無線LANの実現に向けては様々なMAC(Media Access Control)プロトコルが提案されている~\cite{fdmac, goyal, janus, fdmac2, morif, fdmac3, fcts}.
	\par
	全二重通信無線LANでは図\ref{fig:topology}に示すような従来の半二重通信では存在しなかった二つの干渉が問題となる.
	一つは,送受信を行っているAP(Access Point)において,送信信号が所望の受信信号に干渉を及ぼす自己干渉であり,
	もう一つは,STA(station) $j$の送信信号がもう一方のSTA $i$の受信信号に干渉を及ぼすユーザ間干渉である.
	自己干渉は自己干渉除去技術によって無視できるレベルまで削減できることが示されている~\cite{fdmac, stanford1}.
	また,ユーザ間干渉に関しても,適切なSTAの組み合わせを選び出すことや,
	送信電力制御を行うことでユーザ間干渉を削減する手法が提案されている~\cite{contra, promac}.
	しかし,~\cite{contra}では1対のSTAの組み合わせを選び出すために過去の全二重通信の成功確率が用いられるため,
	干渉の影響のみが考慮されSTA間の公平性が低下する問題がある.
	また~\cite{promac}ではSTAの組み合わせの決定に際してCSMA/CA(Carrier Sence Multiple Access with Collision Avoidance)のバックオフアルゴリズムを用いるためある程度の公平性は保証されるが,
	バックオフカウンタがスループットに依存する手法をとっているため干渉の影響が依然として大きく,公平性は低い.
	さらに,両者は各STAが利用しているアプリケーションサービスの要求の違いが考慮されておらず,
	例えば低遅延を要求するようなSTAが存在してもそれに応えることはできない.
	これらの理由から,STAの大きくQoSが低下してしまう問題がある.
	\par
	本稿では,これらの問題を解決するために,まずSTA間の送信機会の公平性を改善する手法を提案する.
	次に,各STAが利用しているアプリケーションサービスの要求に違いがある場合,特に低遅延を要求するSTAとそうでないSTAが混在している場合において,
	低遅延を要求するSTAの送信機会を増加させQoSを向上する手法を提案する.
	さらに,二つの提案手法の有効性を計算機シミュレーションによって評価する.
	\par
	本稿の構成は以下のとおりである.第2章で本稿で扱うシステムモデルについて述べ,第3章では従来のMACプロトコル~\cite{promac}について述べる.
	さらに,第4章において提案手法について述べ,第5章では提案手法の有効性をシミュレーションによって評価する.
	最後に第6章でまとめとする.

	\begin{figure}[t]
		\centering
		\epsfig{file=fig/model_relay.eps, scale=0.4}
		\caption{一方向全二重通信}
		\label{fig:topology}
	\end{figure}

\section{システムモデル}
	\begin{figure}[t]
		\centering
		\epsfig{file=fig/pos.eps, scale=0.6}
		\caption{システムモデル}
		\label{fig:model}
	\end{figure}

	本稿で検討するシステムモデルを図\ref{fig:model}に示す.
	1台のAPが$L$四方の領域の中心に設置され,その周りに$N$台のSTAがランダムに配置されているとする.
	それらSTAのインデックス集合を$\mN=\{1,2,...,N\}$とする.
	この$N$台のSTAの中から,図\ref{fig:topology}のようにAPからの下り通信を受信するSTA $i$と,
	APへの上り通信を行うSTA $j$を選び出す.
	このとき,STAの組み合わせを$\sij$と表現し,$i,\ j \in \{0\}\cup \mN$であり,
	STAは自己干渉除去技術を持たないため双方向全二重通信はできないと仮定して$i\neq j$とする.
	ただし,$i=0$のときは下り通信を伴わない上り通信のみの半二重通信であり,
	$j=0$のときは上り通信を伴わない下り通信のみの半二重通信であるとする.
	さらに,この$N$台のSTAの中に低遅延を要求するアプリケーションサービスを用いているSTAが$n$台存在し,
	そのインデックス集合を${\mathcal D}\subset\mN$とする.
	ただし,低遅延を要求しないSTAのインデックス集合は${\overline {\mathcal D}}$と表し,
	${\mathcal D} \cup {\overline {\mathcal D}}=\mN,\ {\mathcal D} \cap {\overline {\mathcal D}}=\phi$とする.

\section{従来方式}
	本章では従来方式~\cite{promac}のMACプロトコルについて述べる.
	このMACプロトコルはシステムスループットを最大化するため,
	各STAの組み合わせごとに推定されるスループットを評価関数に用いた最適化問題を解くことで
	その組み合わせによって全二重通信が行われる確率を求め,
	その確率に基づいてSTAの組み合わせが決定される.
	\subsection{STAの組み合わせの決定手順}\label{sec:promac}
		本節では~\cite{promac}におけるSTAの組み合わせの決定手順について述べる.
		このMACプロトコルでは,APとSTA $i$,$j$の組み合わせで全二重通信が行われる確率$\pij$に基づいてSTA $i$,$j$が決定される.
		\par
		まず,全二重通信の組み合わせを式\eqref{eq:cfull}に示す.
		\begin{equation}
			{\mathcal C}_{\rm full} \triangleq \{\sij : i,j\in{\mathcal N},\ i\neq j,\ r^{\sij}_{\rm d},\ r^{\sij}_{\rm u}>\epsilon\} \label{eq:cfull}
		\end{equation}
		ただし,$r_{\rm d}^{\sij}$,$r_{\rm u}^{\sij}$はそれぞれAPからSTA $i$への下りの実効スループット,
		STA $j$からAPへの上りの実効スループットであり,$\epsilon$はスループットが0に近くなるようなSTAの組み合わせを除くためのしきい値である.
		${\mathcal C}_{\rm full}$に対して,上り下りそれぞれの実効スループット$\rd$,$\ru$を推定し,その合計を$\rij$とする.
		さらに,半二重通信の組み合わせ
		\begin{equation}
			{\mathcal C}_{\rm half} \triangleq \{\sij : i=0\ {\rm or}\ j=0,\ \rij >\epsilon\}
		\end{equation}
		に対しても実効スループット$\rij$を推定する.
		得られた$\rij$に基づいて以下の最適化問題を解き,確率$\pij$を得る.
		\begin{align}
			&{\mathcal P}_1: && {\rm max} \sum_{(i,\ j)\in{\mathcal C}} p^{(i,\ j)}r^{(i,\ j)} &&&&&& \label{eq:p1}\\
			&{\rm subject\ to} && \sum_{j\in\{j:(i,\ j)\in{\mathcal C}\}} p^{(i,\ j)} \geq \eta_{\rm d}^{(i)},\ \forall i\in {\mathcal N}  \\
			&&& \sum_{i\in\{i:(i,\ j)\in{\mathcal C}\}} p^{(i,\ j)} \geq \eta_{\rm u}^{(j)},\ \forall j\in {\mathcal N} \label{eq:pu}\\
			&&& \sum_{j\in\{j:(i,\ j)\in{\mathcal C}\}} p^{(i,\ j)}=1 \\
			&{\rm variables:} &&p^{(i,\ j)} \in {\mathbb R}_{\geq 0},\ \forall(i,\ j)\in {\mathcal C} \nonumber
		\end{align}
		ただし,${\mathcal C} = {\mathcal C}_{\rm full} \cup {\mathcal C}_{\rm half}$である.
		$\eta_{\rm d}^{(i)}$はSTA $i$への下り通信のトラヒックに比例した値であり,
		STA $i$が下り通信の送信先となる確率$p_{\rm d}^{(i)}=\sum_{j\in\{j:(i,\ j)\in{\mathcal C}\}} p^{(i,\ j)}$
		の最低値を定める役割を果たす.
		同様に,$\eta_{\rm u}^{(j)}$はSTA $j$が持つ上り通信のトラヒックに比例した値であり,
		STA $j$が上り通信の送信権を得る確率$p_{\rm u}^{(i)}=\sum_{j\in\{i:(i,\ j)\in{\mathcal C}\}} p^{(i,\ j)}$
		の最低値を定める役割を果たす.
		また,以下の条件が満たされるとき必ず解が得られることが示されている.
		\begin{align}
			r_{\rm d}^{(i,\ 0)} &>\epsilon,\ \forall i\in\mN \\
			r_{\rm u}^{(0,\ j)} &>\epsilon,\ \forall j\in\mN \\
			\sum_{i\in\mN}\eta_{\rm d}^{(i)} + \sum_{j\in\mN}\eta_{\rm u}^{(j)} &=1 \label{eq:feasible}
		\end{align}
		なお,この最適化問題は最大でAPが送信するビーコン信号周期ごとに解かれ,更新された確率$\pij$はビーコンフレームによってSTAに通知される.
		\par
		次に,得られた$\pij$を用いてSTA $i$,$j$を決定する方法を述べる.
		APは
		\begin{equation}
			p_{\rm d}^{(i)}= \sum_{j\in\{j:(i,\ j)\in{\mathcal C}\}}p^{(i,\ j)},\ \forall i \in \{0\}\cup{\mathcal N}
		\end{equation}
		によって各STAが下り通信の送信先となる確率$p_{\rm d}^{(i)}$を求め,この確率に従って一様ランダムに送信先STA $i$を選択する.
		次にSTA $j$を決定する.
		各STAは,「APがSTA $i$へ下り通信を行うことが決まった上で自身がAPへの上り通信の送信権を獲得する」という条件付き確率$p_{\rm u}^{\sij}$を
		\begin{equation}
			p_{\rm u}^{\sij} = P(j\ {\rm wins\ uplink|AP\ sends\ to}\ i)=\pij / p_{\rm d}^{(i)}
		\end{equation}
		と求め,コンテンションウィンドウサイズ${\rm CW}^{\sij}_{\rm u}$を
		\begin{equation}
			{\rm CW}^{\sij}_{\rm u} = \lceil 1/p_{\rm u}^{(j)} \rceil
		\end{equation}
		とする.
		各STAは$[0,\ {\rm CW}^{\sij}_{\rm u}]$の一様分布から生成されるバックオフカウンタ$w_{\rm u}^{\sij}$を設定し,
		CSMA/CAのバックオフアルゴリズムを用い,バックオフカウンタを1ずつ減らし,
		その結果,最初にカウンタが0となったSTAがSTA $j$として上り通信の送信権を獲得する.

	\subsection{問題点}
		前節で述べたように~\cite{promac}では,STA $i$,$j$の干渉が小さい組み合わせが選ばれやすい.
		図\ref{fig:numtx}にSTA台数を$N=50$としたシステムで
		従来のSTAの組み合わせの決定手順を再現したシミュレーションによって得られた各STAの上り通信送信回数を示す.
		結果からわかるように,一部のSTAが非常に選ばれやすいことがわかる,
		このことから,送信機会に関する公平性は低く,加えて,低遅延を要求するようなアプリケーションサービスを利用しているSTAが存在する場合にも,
		その要求に応えることはできないという問題がある.
		本稿では,この2つの問題点に関して解決を図り,QoSの向上を目指す.

		\begin{figure}[t]
			\centering
			\epsfig{file=graph/numtx.eps, scale=0.6}
			\caption{従来方式によるSTAの送信回数の分布}
			\label{fig:numtx}
		\end{figure}

\section{提案方式}
\subsection{公平性の改善}\label{sec:fair}
	本節ではSTA間の送信機会の公平性を改善するための検討を行う.
	最初に述べたように従来のMACプロトコルでは干渉が小さい組み合わせが選ばれやすく,
	STA間の公平性が低下するという問題点がある.
	この問題を解決をするため以下のように,式\eqref{eq:p1}に待機時間$d^{(j)}$の項を評価関数に追加する.
	\begin{equation}
		{\mathcal P}_2: {\rm max} \sum_{(i,\ j)\in{\mathcal C}} p^{(i,\ j)}r^{(i,\ j)}(d^{(j)})^{\alpha} \label{eq:p2}
	\end{equation}
	待機時間$d^{(j)}$とはSTA $j$のバッファの先頭にフレームが到着してから現在時刻までの時間である.
	この待機時間の項を追加することで,待機時間が長いSTAを含んだ組み合わせが選ばれる確率が高くなり,
	送信機会を得られていないSTAほど送信機会を得やすくなる.
	また,待機時間は飽和トラヒックである限りは前回の送信時刻からの経過時間と同じであるため,
	新たに各STAの待機時間情報を収集する必要はなく,APが各STAごとに最新の送信時刻を記憶していれば,
	現在時刻との差として得られる.
	追加する項として各STAの平均送信間隔や送信回数そのものを選択しない理由は,
	両者はいずれも積算値であるため,瞬時性がなく,新たに追加されたSTAに対応できないためである.
	\par
	公平性の改善を行うと,公平性の改善を行わない場合に比べて比較的干渉の多いSTAの組み合わせが選ばれることが多くなり,
	システムスループットの低下が考えられる.
	そのため,公平性の改善とシステムスループットの低下のトレードオフを調整可能とするための重み係数$\alpha\geq 0$を導入している.
\subsection{低遅延を要求するSTAのQoSの向上}
	本節では,システム全体の公平性を改善した上で,さらに低遅延を要求するSTAのQoS改善を行う提案方式について述べる.
	低遅延を要求するSTAのQoSを向上させるためには,
	上り通信を行う確率$p_{\rm u}^{(j)}$を大きくし,送信機会を増加させればよい.
	これを実現するために,式\eqref{eq:pu}において$p_{\rm u}^{(j)}$の最低値を決定していた$\etau$を大きくすることを考える.
	具体的には,以下の式のように低遅延を要求していないSTAの最低送信確率$\etau$から$x_j$だけ譲り受け,
	それを低遅延を要求するSTAの最低送信確率${\hat \eta}_{\rm u}^{(j)}$に加える.
	\begin{align}
		{\hat \eta}_{\rm u}^{(j)} &= \etau -x_j > 0,\ \forall j \in {\overline {\mathcal D}} \label{eq:etadbar}\\
		{\hat \eta}_{\rm u}^{(j)} &= \etau + x_j, \ \forall j \in {\mathcal D} \label{eq:etad}\\
		\sum_{j\in{\overline {\mathcal D}}} x_j &= \sum_{j\in{\mathcal D}} x_j \label{eq:sub}
	\end{align}
	なお,式\eqref{eq:sub}は式\eqref{eq:feasible}を満たし可解性を失わないための条件である.
	提案方式では以上のように新たに設定された${\hat \eta}_{\rm u}^{(j)}$を最適化問題の制約条件である式\eqref{eq:pu}に用いる.
	これによって,低遅延を要求するSTAの送信機会が増加し送信間隔が短くなることで遅延時間が短縮されQoSが改善される.

\section{シミュレーション評価}
	本章では提案手法の有効性をシミュレーションによって評価する.
	シミュレーション諸元を表\ref{tab:param}に示す.
	MACプロトコルは~\cite{promac}に従い,上下通信ともに飽和トラヒックの場合を取り扱う.

	\begin{table}[t]
		\centering
		\caption{シミュレーション諸元}
		\label{tab:param}
		\begin{tabular}{cc} \hline
			領域の大きさ $L$ & 100\,m \\
			伝送速度 & シャノン容量 \\
			送信電力 & 15\,dBm \\
			雑音指数 & 10\,dB \\
			減衰定数 & 3 \\
			周波数帯 & 5\,GHz \\
			自己干渉キャンセル & 110\,dB \\
			シミュレーション時間 & 10\,s \\\hline
		\end{tabular}
	\end{table}

	\subsection{公平性の改善}
	図\ref{fig:fair}にSTA台数を$N=50$とした場合の各STAの全シミュレーション時間内での上り通信送信回数を示す.
	図\ref{fig:numtx}と比較して,一部のSTAが極端に選ばれやすいという現象が改善されていることがわかる.
	次に,システムスループットと公平性の関係性を確認するため,
	図\ref{fig:thr_fair}にシステムスループットとSTA間の公平性を示す.
	ただし,いずれも図\ref{fig:model}のような10種のSTA配置の平均値であり,
	公平性の評価はJain's fairness index~\cite{jain}に各STAの送信回数を代入することで求めている.
	STA間の送信機会に関する公平性を大きく改善した上,
	重み係数$\alpha$を変化させることで,公平性の改善とシステムスループット低下のトレードオフをシステムの要求に応じて変化可能である.
	また,この場合$\alpha=0.3$のときにスループットの低下が小さく,公平性が高くなっている.
	\par
	次に,このスループットの低下が小さく,公平性が高くなる$\alpha$の値がシミュレーション条件によってどのように変化するかを示す.
	図\ref{fig:chgparam}\subref{fig:chgtopology}に最初の10種とは異なるあるSTA配置の場合のシステムスループットとSTA間の公平性を示す.
	この場合,システムスループットの低下が小さく,STA間の公平性が高い$\alpha$の値は0.2程度である.
	さらに,図\ref{fig:chgparam}\subref{fig:chgnum}にSTA台数を$N=30$とした場合の結果を示す.
	ただし,結果は10種のSTA配置による結果の平均値である.
	この場合にも,$\alpha=0.2$あたりでバランスが良いことがわかる.
	いずれもスループットの値やfairness indexの値は異なるが,概ね$\alpha$は0.2から0.3で公平性の改善とスループットの低下のバランスが良いことがわかる.

	\begin{figure}[t]
		\centering
		\epsfig{file=graph/numtxfair.eps, scale=0.6}
		\caption{提案方式によるSTAの上り通信送信回数の分布}
		\label{fig:fair}
	\end{figure}

	\begin{figure}[t]
		\centering
		\epsfig{file=graph/thr_fair.eps, scale=0.6}
		\caption{重み係数$\alpha$に対するシステムスループットとfairness index}
		\label{fig:thr_fair}
	\end{figure}

	\begin{figure}[t]
		\centering
		\subfloat[あるSTA配置におけるシステムスループットとfariness index]{
			\epsfig{file=graph/chgtopology.eps, scale=0.6}
			\label{fig:chgtopology}
		}
		\\
		\subfloat[STA台数を$N=30$に変更した場合のシステムスループットとfairness index]{
			\epsfig{file=graph/chgnum.eps, scale=0.6}
			\label{fig:chgnum}
		}
		\caption{重み係数$\alpha$に対するシステムスループットとfairness index(シミュレーション条件を変更した場合)}
		\label{fig:chgparam}
	\end{figure}

	\subsection{低遅延を要求するSTAのQoSの向上}
	次に,低遅延を要求するSTAのQoS向上について評価を行う.
	本シミュレーションでは式\eqref{eq:etadbar}における$x_j$は
	低遅延を要求しない全STA共通の値$x_j=x,\ \forall j\in {\overline {\mathcal D}}$とし,
	式\eqref{eq:etad}における$x_j$も低遅延を要求する全STA共通の値$x_j=x|{\overline {\mathcal D}}|/|{\mathcal D}|,\ \forall j \in {\mathcal D}$とした.ただし,$|{\overline {\mathcal D}}|$は${\overline {\mathcal D}}$の,
	$|{\mathcal D}|$は${\mathcal D}$の要素数を表す.
	また,STA台数$N$は50台であり,$\alpha=0.3$である.
	\par
	まず最初に低遅延を要求するSTA台数を$n=5$,$x=0.05$とした場合に,それらの送信間隔が短縮されていることを確認する.
	低遅延を要求するSTAは${\mathcal D}=\{46,\ 47,\ 48,\ 49,\ 50\}$であるとする.
	図\ref{fig:inter}\subref{fig:interfair}に公平性のみを考慮した場合について,
	図\ref{fig:inter}\subref{fig:intereta}に低遅延を要求するSTAの送信機会を増加させた場合について各STAの平均送信間隔を示す.
	公平性のみを考慮した場合は全STA間の送信機会の公平性が高いことから,
	送信間隔のばらつきが少ないが,低遅延を要求するSTA 46から50の送信間隔も平均43\,msと長い.
	一方,アプリケーションサービスの違いを考慮し,低遅延を要求するSTAの送信機会を増加させたところ,
	送信間隔を15\,msと1/3程度まで削減することができた.
	しかし,低遅延を要求しないSTA間の公平性は,公平性のみを考慮していた場合に比べて劣化した.

	\begin{figure}[t]
		\centering
		\subfloat[公平性の改善のみを行った場合のSTAの平均送信間隔]{
			\epsfig{file=graph/interfair.eps, scale=0.6}
			\label{fig:interfair}
		}
		\\
		\subfloat[低遅延を要求するSTAの送信機会を増加させた場合のSTAの平均送信感覚]{
			\epsfig{file=graph/intereta.eps, scale=0.6}
			\label{fig:intereta}
		}
		\caption{STAの平均送信間隔の比較}
		\label{fig:inter}
	\end{figure}

\section{まとめ}
本稿では,全二重通信無線LANの既存のMACプロトコルが持つSTA間の公平性の低さと
STAが利用しているアプリケーションサービスの違いを考慮できていないといった問題について検討し,
それらを解決する手法を提案した.
各STAの待機時間を考慮することで,送信機会を得ることができていないSTAに送信機会を与えることで
STA間の公平性を大幅に改善し,さらに,重み係数$\alpha$によって公平性の改善とスループットの低下のバランスを調整できることを示した.
また,低遅延を要求するようなSTAの送信機会を増加させることでQoSの改善を行った.



\bibliographystyle{sieicej}
\bibliography{main2}

\end{document}
